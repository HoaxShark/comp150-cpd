% Please do not change the document class
\documentclass{scrartcl}

% Please do not change these packages
\usepackage[hidelinks]{hyperref}
\usepackage[none]{hyphenat}
\usepackage{setspace}
\doublespace

% You may add additional packages here
\usepackage{amsmath}

% Please include a clear, concise, and descriptive title
\title{Personal Reflection}

% Please do not change the subtitle
\subtitle{COMP150 - CPD Report}

% Please put your student number in the author field
\author{1706966}

\begin{document}

\maketitle

\section{Introduction}

Throughout this paper I will be reflecting on certain aspects I have highlighted over my first 13 weeks at Falmouth University. The skills I have chosen to look at are, idea sharing, coding proficiency, debugging skills, stress management and meeting planning when working in a group. All the skills will be looked at while considering my career goals, currently these are to achieve a level of proficiency in software development to be able to work with a company or even do freelance work.

\section{Interpersonal - Idea Sharing}

My ability to share ideas successfully to a group is necessary if I am to work as part of a team later in my career, the early outcomes of trying to share my ideas at university were disappointing. The first group I had didn't take on any of my ideas, I fell back from trying to push forward ideas as it seemed like wasted effort. Throughout the 13 weeks this has improved, partly due to more accepting groups and my determination to stick with selling them an idea when I consider it important, finding the balance between being dedicated to an idea and letting it go is tricky but I believe I’ve improved over the course so far. I've come up with a SMART goal to help me improve further.
\subsection{SMART}
I will make sure to have at least 2 ideas used for any of our group work projects in the upcoming 6 weeks. This will provide me with good experience in sharing my ideas in group projects for when I work in software development.


\section{Cognitive - Coding Proficiency}
I knew coming into the course that my general knowledge of coding was low, and it has been obvious to me when working with others in groups that many people have a higher understanding than I do, although I pick things up quickly it would be extremely useful to keep putting in extra work to improve my skills in this area. Coding proficiency is essential to where I want to go with my career, it is the main aspect needed. I have already been putting in effort to learn more but with a set SMART goal I will be able to measure and view my success. 
\subsection{SMART}
I will dedicate 3 hours a week for 12 weeks to going through game development training courses of relevant languages on the internet. Providing me with a better understanding of coding, increasing my general knowledge on the subject.


\section{Procedural - Debugging Skills}
Many times during my weeks at university I have spent hours looking over errors with my code, this is a common part of the work I'm looking to go into and knowing how to do it more efficiently will be greatly beneficial. My ability to debug has improved over the weeks mostly just by learning more and more about how code works and interacts with itself, but also by learning how to use the tools within IDE's. The more I can learn about ways to debug the more time and stress I can save myself later when working, therefore I have set a SMART goal to help focus on improving this skill.
\subsection{SMART}
I will spend 2 hours a week for 9 weeks researching debugging methods for the language I am currently learning. Allowing me to problem solve my own errors quicker giving a faster turnaround on my work.


\section{Affective - Stress Management}
I don't often get stressed in my daily life and it's something that I'm learning to deal with moving into this line of work. I have found myself a few times stuck unable to figure out why something isn't working, or how to get the solution needed. Each time spending hours slowly getting more annoyed. Being stressed doesn't help with the work and lends itself to an unhealthy lifestyle, so finding ways to handle stress is imperative. I've heard that taking small breaks when stuck on an issue can allow you to relax and often find a way to work out the problem once you get back and try again. So as for the other skills I have a SMART goal to help.
\subsection{SMART}
I will make sure to take 15 minute breaks if i'm starting to stress over a problem for more than an hour. This will let me relax and take a look at the problem with a fresh mind.


\section{Dispositional - Meeting Planning}
During big group projects I haven't found a problem organising times to meet and work with people, however for the smaller groups of 2-3 I have found myself not pushing to create meeting times. I'm not sure if its due to the empathise on only a couple of us to make these arrangements. Whatever the reason it became problematic when working on these small projects, I actually set myself the SMART goal below a while back and it really helped kick myself into planning with my pair and working towards the project. This has shown me that just by setting these small goals for yourself it can drive you slightly more than you usually might to achieve the goal set.
\subsection{SMART}
I will organise one session a week with my paired programming partner to work on the tinkering audio project until its completion. Giving us a chance to hit our 3 commits per week.

\section{Conclusion}
In conclusion, I feel that I have picked important skills to focus on, if I keep to the SMART goals I have set then I will be in a better place to be moving into a career in this industry later in life and they will also help with my upcoming projects. The goals set I believe are not too far-fetched and I look forward to trying to complete them.


\end{document}
