% Please do not change the document class
\documentclass{scrartcl}

% Please do not change these packages
\usepackage[hidelinks]{hyperref}
\usepackage[none]{hyphenat}
\usepackage{setspace}
\doublespace

% You may add additional packages here
\usepackage{amsmath}

% Please include a clear, concise, and descriptive title
\title{Personal Reflection}

% Please do not change the subtitle
\subtitle{COMP130 - CPD Report}

% Please put your student number in the author field
\author{1706966}

\begin{document}

\maketitle

\section{Introduction}

This paper will be used to reflect on my current progression in regards to my course at Falmouth University. I will take a look at 5 key skills that I have identified as needing improvement, go into more detail as to why these skills are important and set myself related SMART goals to follow over the summer break. Skills I have chosen to look at are the following; Group organisation, logic problem solving, referencing vs pointers, group dedication and working early. Improving these skills will put me in a better stead to reach my career goals, currently these are to achieve a level of proficiency in software development to be able too work with a company or even do freelance work.

\section{Interpersonal - Group Organisation}

I found during the second term that my group team lacked organisation and drive to meet up and complete the project, we had a scrum master who didn't organise anything for us. I had the ability to take more control here and plan things myself but decided to take a step back and not enforce meetings on the team. The lack of meetings and working together felt like it really hindered our progress towards making a cohesive game together. Having a good handle on this group organisation skills will put me in good stead for working within an organisation, by giving me the ability to make sure any team that I work in have good opportunities to meet up, communicate their ideas and current work positions. Below is a SMART goal for me to use to improve this skill.
\subsection{SMART}
During the next group project I will strongly encourage that the team book and meet for daily stand ups throughout the length of the project. This will give me good experience with subtly  organising groups to better improve the end products.


\section{Cognitive - Referencing vs Pointers}
My understanding of code and syntax is slowly improving, but an area in which I am currently weak on is knowing when best to use references or pointers. I need to improve my understanding of the differences between them and how best to use them. This is an important part of being able to write successful and efficient code, which will be important for me when looking for a career in software development. I have come up with a SMART goal to help push myself to learn more about this skill.
\subsection{SMART}
I will dedicate 2 hours a week for 12 weeks to researching pointers and referencing. I will spend some of that time practising the use of them by writing code snippets. This will allow me to write better, more efficient code in the future which is important if wanting to work in the software industry.


\section{Procedural - Logic Problem Solving}
As my understanding of how code is written has improved I have found myself wanting to know more efficient ways of solving problems. The way I currently work out most logic problems solve the problem, but i'm sure there are more efficient ways of doing so. Being able to write code that runs efficiently is a very important aspect in software development and having this skill will greatly improve my employability with companies. Therefore I have set a SMART goal to help focus on improving this skill.

\subsection{SMART}
I will spend 1 hour a day, 5 days a week for 12 weeks working on code wars challenges on-line. I will make sure to look at the most efficient solutions from other participants after completing a challenge. This will give me a chance to work out problems and then see some of the best ways it has been done, giving me a good insight into efficient logic problem solving.


\section{Affective - Group Dedication}
During this group project I found that the lack of interest and dedication shown by the rest of the team for the project ended up rubbing off on me as well. This not only let my personal input to the project slip below what I consider to be satisfactory. This annoyed me and would not be acceptable working practice in a paid environment. As for the other skills I have a SMART goal to help avoid this issue in the future.
\subsection{SMART}
Over the course of my next group project I will work for at least 20 hours on the project over the course of the week. Having this goal will mean no matter the dedication of the team I am still putting in a satisfactory effort in to the group project.


\section{Dispositional - Working Early}
I have found myself more often than not leaving work towards the end of a deadline and picking it up to do just before, although I can work in this manner I don't believe it allows me to get the best results from my work. With some projects I would have liked to have had time to change some smaller aspects of the work and had I started sooner I would have had the time. When working for a company it will not be acceptable to be lax with your time and projects will be expected to be worked on from start to finish date. 
\subsection{SMART}
I will take a look at assignment briefs for work when assigned and use a calendar system to organise my work time efficiently from start to end of the deadline. This will help with my personal organisation and hopefully allow me to produce better end products.

\section{Conclusion}
In conclusion, the skills that I have picked to improve on will help later when working in the industry but also help me through the summer break and throughout my remaining 2 years at university. I believe the SMART goals I have set are reasonable and will help facilitate my learning and optimise my time at university. I look forward to seeing myself improve because of them.


\end{document}
